% Kale Ewasiuk (kalekje@gmail.com)
% +REVDATE+
% Copyright (C) 2021 Kale Ewasiuk
%
% Permission is hereby granted, free of charge, to any person obtaining a copy
% of this software and associated documentation files (the "Software"), to deal
% in the Software without restriction, including without limitation the rights
% to use, copy, modify, merge, publish, distribute, sublicense, and/or sell
% copies of the Software, and to permit persons to whom the Software is
% furnished to do so, subject to the following conditions:
%
% The above copyright notice and this permission notice shall be included in
% all copies or substantial portions of the Software.
%
% THE SOFTWARE IS PROVIDED "AS IS", WITHOUT WARRANTY OF
% ANY KIND, EXPRESS OR IMPLIED, INCLUDING BUT NOT LIMITED
% TO THE WARRANTIES OF MERCHANTABILITY, FITNESS FOR A
% PARTICULAR PURPOSE AND NONINFRINGEMENT.  IN NO EVENT
% SHALL THE AUTHORS OR COPYRIGHT HOLDERS BE LIABLE FOR
% ANY CLAIM, DAMAGES OR OTHER LIABILITY, WHETHER IN AN
% ACTION OF CONTRACT, TORT OR OTHERWISE, ARISING FROM,
% OUT OF OR IN CONNECTION WITH THE SOFTWARE OR THE USE
% OR OTHER DEALINGS IN THE SOFTWARE.


\documentclass[11pt,parskip=half]{scrartcl}
\setlength{\parindent}{0ex}
\newcommand{\llcmd}[1]{\leavevmode\llap{\texttt{\detokenize{#1}\ }}}
\newcommand{\cmd}[1]{\texttt{\detokenize{#1}}}
\newcommand{\qcmd}[1]{``\cmd{#1}''}
\usepackage{url}
\usepackage{showexpl}
\lstset{explpreset={justification=\raggedright,pos=r,wide=true}}
\setlength\ResultBoxRule{0mm}
\lstset{
	language=[LaTeX]TeX,
	basicstyle=\ttfamily\small,
	commentstyle=\ttfamily\small\color{gray},
	frame=none,
	numbers=left,
	numberstyle=\ttfamily\small\color{gray},
	prebreak=\raisebox{0ex}[0ex][0ex]{\color{gray}\ensuremath{\hookleftarrow}},
	extendedchars=true,
	breaklines=true,
	tabsize=4,
}
\addtokomafont{title}{\raggedright}
\addtokomafont{author}{\raggedright}
\addtokomafont{date}{\raggedright}
\author{Kale Ewasiuk (\url{kalekje@gmail.com})}
\usepackage[yyyymmdd]{datetime}\renewcommand{\dateseparator}{--}
\date{\today}





\newcommand{\tMC}{\cmd{\\MC}}

\RequirePackage{lutabulartools}
\title{lutabulartools}
\subtitle{some useful tabular tools (LuaLaTeX-based)}

\begin{document}
\maketitle

lutabulartools is a package that contains a few useful macros to help with tables.
Most functions require LuaLaTeX. The following packages are loaded:
\texttt{
{booktabs},
{multirow},
{makecell},
{xparse},
{array},
{xcolor},
{colortbl},
{luacode},
{penlight},
}

\section{\texttt{\textbackslash MC} -- Magic Cell}
\texttt{\textbackslash MC} (magic cell) combines the facilities of
\cmd{\multirow} and \cmd{\multicolumn} from the \texttt{multirow} package, and \cmd{\makcell} from the titular package.
With the help of LuaLaTeX, it takes an easy-to-use cell specification and employs said commands as required.
Here is the usage:

\texttt{\textbackslash MC * [cell spec] <cell format> (override multicolumn col) \{contents\} }

\llcmd{*}This will wrap the entire command in \{\}. This is necessary for \texttt{siunitx} single-column width columns.
However, the \tMC\  command attempts to detect this automatically.

\llcmd{[cell spec]}%
Any letters placed in this argument are used for cell alignment.
You can use one of three: \qcmd{t}, \qcmd{m}, \qcmd{b} for top, middle, bottom (vertical alignment),
or \qcmd{l}, \qcmd{c}, \qcmd{r} for horizontal alignment.
By default, \tMC\   will try to autodetect the horizontal alignment based on the current column.
If it can't, it will be left-aligned.
The default vertical alginment is top.

This argument can also contain two integers, separated by a comma (if two are used).
\qcmd{C,R}, \qcmd{C}, or \qcmd{,R} are a valid inputs,
where R=rows (int), and C=columns, (int).
If you want a 1 column wide, multirow cell,
you can pass \qcmd{,R}. These numbers can be negative.
If no spec is passed, (argument empty), \tMC\
acts like a \texttt{makecell}.
Additionally, you can pass \qcmd{+} in place of C (number of columns wide),
and it will make the cell width fill until the end of the current row.

Examples:\\
\qcmd{\MC[2,2]} means two columns wide, two rows tall.\\
\qcmd{\MC[2,1]} or \qcmd{\MC[2]} or  means two columns wide, one row tall.\\
\qcmd{\MC[1,2]} or \qcmd{\MC[,2]} means one column wide, two rows tall.\\
In any of these examples, you can place the alignment letters anywhere.


\llcmd{(override}%
You may want to adjust the column specification of a multicolumn  cell,\\
\llcmd{multicolumn)}for example, using
\texttt{@\{\}c@\{\}} to remove padding between the cell.

\llcmd{<cell format>}%
You can place formatting like \cmd{\bfseries} here.

Here's an example.

\begin{LTXexample}
\begin{tabular}{| c | c | c | c | c | c |}\toprule
 \MC[2,2cm]<\ttfamily>{2,2cm}   & \MC[2r]<\ttfamily>{2r} & 5 & \MC[,2b]<\ttfamily>{,2b}\\
   &   & 3 & 4 & 5 & \\\midrule
 1 & 2 & \MC[2l](@{}l)<\ttfamily>{2l (\@\{\}l)} & 5 & 6666\\\cmidrule{3-4}
 1 & \MC[+r]<\ttfamily>{+r}  \\
  \\
 1 & 2 & 3 & 4 & 5 & \MC[,-2]<\ttfamily>{,\\-2}\\
\end{tabular}
\end{LTXexample}


%%% https://tex.stackexchange.com/questions/287346/width-of-column-after-multicolumn-header

%\textbackslash MC[<spec>]{<contents}

\subsection{Notes}
This package redefines the {\ttfamily tabular} and {\ttfamily tabular*} environments.
It uses Lua pattern matching to parse the column specification of the table to know how many columns there are,
and what the current column type is. If you have defined a column that creates many, it will not work.
This will be worked out in later package revisions.
%But, to know how many columns there are you, you will have
%either have to manually set todo the lua variable before the table, or,
%in the last column of the first row, \texttt{get\_tab\_colnum ()}.
%The above Lua function can be used to get the current column number.
%You will also remove the ability for this package to know what the current column spec is



\section{Some additional rules}
This package also redefines the \texttt{booktabs} midrules.\\
\llcmd{\gmidrule}is a full gray midrule.\\
By taking advantage of knowing how many columns there are (if you chose to redefine \texttt{tabular}),
you can specify individual column numbers (for a one column wide rule),
or reference with respect to the last column (blank, \texttt{+1}, \texttt{+0}, or \texttt{+} means last column,
\texttt{+2} means second last column, for example)
or omit the last number.\\
\llcmd{\cmidrule}is a single partial rule, with the above features\\
\llcmd{\gcmidrule}is a single partial gray rule, with the above features\\
You can add multiple \qcmd{cmidrule}'s with the \texttt{(g)cmidrules} command. Separate with a comma.
You can apply global trimming of the rules with the \qcmd{()} optional argument, and then
override it for a specific rule by placing \qcmd{r} or \qcmd{l} with the span specification.\\
\llcmd{\gcmidrules}Can produce multiple, light gray partial rules\\
\llcmd{\cmidrules}Can produce multiple black partial rules.\\
Here's an example:

\begin{LTXexample}
\begin{tabular}{c c c c c c}\toprule
 1 & 2 & 3 & 4 & 5 & 6\\ \cmidrule{+1}  % rule on last column
 1 & 2 & 3 & 4 & 5 & 6\\ \cmidrules{1,3-+3,+} % rule on first col, third to third last col, and last col
 1 & 2 & 3 & 4 & 5 & 6\\ \cmidrules{1,3-+3rl,+} % same as above, but trim middle
 1 & 2 & 3 & 4 & 5 & 6\\ \cmidrules(l){1,r3-+3,+1}% trim left for all, but only trim right for middle rule
\end{tabular}
\end{LTXexample}


\subsection{Midrule every X\textsuperscript{th} row}

\llcmd{\midruleX}%
With this command, you can place a rule
every X rows.
You can change the step size and what kind of midrule you prefer.
\begin{verbatim}
\def\midruleXstep{5}
\def\midruleXrule{\gmidrule}
\end{verbatim}


Usage: Insert midrulex at the end of each row using  the column spec.
Before you want counting to begin, you should
apply resetmidruleX (also to avoid header rows).

\begin{LTXexample}
\def\midruleXstep{4}
\def\midruleXrule{\cmidrules{1,3-4}}
\begin{tabular}{rclc@{\midruleX}}
                % ^^^ inject midrule
\toprule
Num  & . & . & .  \\
\midrule\resetmidruleX  % reset
1    & & &  \\
2    & & &  \\
3    & & &  \\
4    & & &  \\
5    & & &  \\
6    & & &  \\
7    & & &  \\
8    & & &  \\
9    & & &  \\
10   & & &  \\
11   & & &  \\
\resetmidruleX % no bottom rule
12   & & &  \\
\bottomrule
\end{tabular}
\end{LTXexample}

\end{document}
